\documentclass[legalpaper,12pt]{article}

% Page set up
\usepackage[margin=2.5cm, top=1.5cm]{geometry}

% Document font and symbols
\usepackage{mathptmx} % Times New Roman
\usepackage{amsmath}
\usepackage{amsfonts,amsthm,amssymb}
\usepackage{esint}

% Language formatting
\usepackage[utf8]{inputenc}
\usepackage[spanish, es-noshorthands]{babel}

% Graph and Drawings
\usepackage[usenames,dvipsnames]{xcolor}

\usepackage{tikz,float}
\usepackage{wrapfig}
\usetikzlibrary{babel}
\usetikzlibrary{patterns}

\usepackage[framemethod=default]{mdframed}
\usepackage{framed}

% Tools
\usepackage{hyperref}
\usepackage{etoolbox,mathtools}
\usepackage[shortlabels]{enumitem}

% Mdframed template
\mdfsetup{innertopmargin=5pt,%
	frametitlealignment=\raggedright,%
	frametitlefont=\texttt,%
	linewidth=2pt,%
	topline=false,rightline=false, bottomline=false,%
	frametitleaboveskip=\dimexpr-1\ht\strutbox\relax,}

%
% Color
%
% -- Frame (print color aside)
\colorlet{TheoremFrameColor}{Salmon}
\colorlet{PropositionFrameColor}{Peach}
\colorlet{CorollaryFrameColor}{Dandelion}
\colorlet{DefinitionFrameColor}{BlueGreen!50!cyan!30}

\colorlet{ProofFrameColor}{black}

\colorlet{ExampleColor}{black}
\colorlet{ExerciseColor}{black}
\colorlet{SolutionColor}{black!60}

% -- Text
\definecolor{TheoremFontColor}{HTML}{000000}
\definecolor{PropositionFontColor}{HTML}{000000}
\definecolor{CorollaryFontColor}{HTML}{000000}
\definecolor{DefinitionFontColor}{HTML}{000000}

% -- Background
\definecolor{TheoremBgColor}{HTML}{FFFFFF}
\definecolor{PropositionBgColor}{HTML}{FFFFFF}
\definecolor{CorollaryBgColor}{HTML}{FFFFFF}
\definecolor{DefinitionBgColor}{HTML}{FFFFFF}

%
% Environment
%

% -- Theorem Env
\newcounter{TheoremCounter}
\newenvironment{Teorema}[1][]{%
	\refstepcounter{TheoremCounter}%
	\mdfsetup{%
		frametitle={%
		\tikz[baseline=(current bounding box.center),outer sep=0pt]
		\node[rectangle,color=TheoremFontColor,fill=TheoremFrameColor]
		{Teorema~\theTheoremCounter\ifstrempty{#1}{.}{:~#1}};},%
		linecolor=TheoremFrameColor,%
		backgroundcolor=TheoremBgColor,
	}%
	\begin{mdframed}[]\relax%
	}{\end{mdframed}}

% -- Proposition Env
\newcounter{PropositionCounter}
\newenvironment{Proposicion}[1][]{%
	\refstepcounter{PropositionCounter}%
	\mdfsetup{%
		frametitle={%
		\tikz[baseline=(current bounding box.center),outer sep=0pt]
		\node[rectangle,color=PropositionFontColor,fill=PropositionFrameColor]
		{Proposición~\thePropositionCounter\ifstrempty{#1}{.}{:~#1}};},%
		linecolor=PropositionFrameColor,%
		backgroundcolor=PropositionBgColor,
	}%
	\begin{mdframed}[]\relax%
	}{\end{mdframed}}

% -- Corollary Env
\newcounter{CorollaryCounter}
\newenvironment{Corolario}[1][]{%
	\refstepcounter{CorollaryCounter}%
	\mdfsetup{%
		frametitle={%
		\tikz[baseline=(current bounding box.center),outer sep=0pt]
		\node[rectangle,color=CorollaryFontColor,fill=CorollaryFrameColor]
		{Corolario~\theCorollaryCounter.\ifstrempty{#1}{}{:~(#1)}};},%
		linecolor=CorollaryFrameColor,%
		backgroundcolor=CorollaryBgColor,
	}%
	\begin{mdframed}[]\relax%
	}{\end{mdframed}}

% -- Definition Env
\newcounter{DefinitionCounter}
\newenvironment{Definicion}[1][]{%
	\refstepcounter{DefinitionCounter}%
	\mdfsetup{%
		frametitle={%
		\tikz[baseline=(current bounding box.center),outer sep=0pt]
		\node[rectangle,color=DefinitionFontColor,fill=DefinitionFrameColor]
		{Definición~\theDefinitionCounter:~#1};},%
		linecolor=DefinitionFrameColor,%
		backgroundcolor=DefinitionBgColor,%
	}%
	\begin{mdframed}[]\relax%
	}{\end{mdframed}}

% -- Proof Env
\newenvironment{Demostracion}[1][]
{%
	{\noindent\ttfamily DEMOSTRACIÓN\ifstrempty{#1}{}{ DE #1}~}
	\fontfamily{cmr}\selectfont
}%
{%
  \newline
  \tikz{\draw (0,0) -- (\linewidth,0) node[above] {\scriptsize\(\blacksquare\)};}
}

% Example Env
\newcounter{ExampleCounter}
\newenvironment{Ejemplo}[1][]%
{
	\refstepcounter{ExampleCounter}%
	\mdfsetup{%
		frametitle={%
		\tikz[baseline=(current bounding box.center),outer sep=0pt]
		\node[rectangle,fill=white]
		{EJEMPLO \theExampleCounter\ifstrempty{#1}{}{:~#1}};},%
		linecolor=ExampleColor,%
		backgroundcolor=white,
	}%
	\begin{mdframed}[]\relax%
	}{\end{mdframed}}

% -- Exercise Env
\newenvironment{Ejercicio}[1][]%
	{%
		\vspace{1ex}
		\textcolor{ExerciseColor}{\hrule height 1pt}
		\vspace{5pt}
		{\noindent\texttt{EJERCICIOS}~}\\[1ex]
	}%
	{%
		\vspace{5pt}
		\textcolor{ExerciseColor}{\hrule height 1pt}
		\vspace{1ex}
	}%

% -- Aligned Cases
\newcommand{\X}{\mathbb{X}}
\newcommand{\Y}{\mathbb{Y}}
\newcommand{\K}{\mathbb{K}}
\newcommand{\Z}{\mathbb{Z}}
\newcommand{\C}{\mathbb{C}}
\newcommand{\R}{\mathbb{R}}
\newcommand{\N}{\mathbb{N}}

\newcommand{\XX}{\mathcal{X}}
\newcommand{\BB}{\mathcal{B}}
\newcommand{\KK}{\mathcal{K}}
\newcommand{\CC}{\mathcal{C}}
\newcommand{\RR}{\mathcal{R}}
\newcommand{\II}{\mathcal{I}}
\newcommand{\FF}{\mathcal{F}}
\newcommand{\MM}{\mathrm{M}}

\DeclareMathOperator{\Rips}{R}
\DeclareMathOperator{\Cech}{\check{C}}
\DeclareMathOperator{\Del}{Del}

\DeclareMathOperator{\argmin}{arg\,min}

\DeclareMathOperator{\centre}{center}
\DeclareMathOperator{\reach}{reach}
\DeclareMathOperator{\dist}{dist}
\DeclareMathOperator{\spn}{span}
\DeclareMathOperator{\supp}{supp}
\DeclareMathOperator{\conv}{conv}
\DeclareMathOperator{\aff}{aff}
\DeclareMathOperator{\card}{card}

% Resize abs and norm
\DeclarePairedDelimiter{\abs}{\lvert}{\rvert}
\DeclarePairedDelimiter{\norm}{\|}{\|}
\makeatletter
\let\oldabs\abs
\def\abs{\@ifstar{\oldabs}{\oldabs*}}
\let\oldnorm\norm
\def\norm{\@ifstar{\oldnorm}{\oldnorm*}}
\makeatother

\usepackage{subfiles}

\title{
Monografía de Pregado\\
``Reconstrucción de Superficies por Nubes de Puntos."\\
}
\author{\normalsize Sebastián Sánchez}
\date{}

\begin{document}

\maketitle

\begin{abstract}
Dada una muestra finita \(\X \subset \R^n\) de una superficie, nos gustaría
desarrollar herramientas que nos permitan reconstruir esta mediante mediante
un complejo simplicial (una triangulación). La principal aplicación de interés 
es el modelaje 3D, no obstante, las herramientas se extienden a otras aplicaciones
tales como reducción de dimensionalidad.
\end{abstract}

\tableofcontents

\newpage

\section{Preliminares}

\subsection{Topología}

Un espacio topológico es una tupla \((X,\tau)\) donde 
\(X\) es un conjunto y \(\tau\) es una colección de subconjuntos de \(X\)
cuyos elementos se dicen abiertos y satisfacen que: \(\varnothing\) y \(X\)
son abiertos, la unión (arbitraria) de abiertos es un abierto y la intersección
finita de abiertos es un abierto. 

Aquí lidiaremos con espacios métricos. Un espacio métrico
es una tupla \((X,d)\) donde \(X\) es un conjunto y \(d\colon X\times X \to \R_{\ge 0}\) es una
función llamada métrica que satisface: \(d(x,y) = 0\) solo cuando \(y = x\), \(d(x,y) = d(y,x)\) y \(d(x,y) \le
d(x,z) + d(z,y)\) para todo trío \(x,y,z\in X\).

Se dice que una colección de conjuntos \(B\) es base de una topología \(\tau\) si todo abierto 
se puede escribir como unión de elementos de \(B\), alternativamente, decimos que \(\tau\) es la 
topología generada por \(B\). Dado un espacio métrico \((X,d)\), \(X\) es un espacio
topológico con la topología generada por las bolas \(B(x,r) \coloneqq \lbrace y\in X\colon d(x,y) < r\rbrace\). 

Una función entre espacios topológicos se dice continua si preimágenes de abiertos son abiertos.
Una función continua biyectiva que tiene inversa continua y biyectiva se dice homeomorfismo (o función
bicontinua). 
Cuando existe un homeomofirsmo entre espacios topológicos \(X\) y \(Y\) denotamos \(X \approx Y\).
Equivalentemente, dos espacios \(X\) y \(Y\) son homeomorfos si existen mapas \(f\colon X \to Y\)
y \(g\colon Y \to X\) continuos tal que \(f\circ g = Id_X\) y \(g\circ f = Id_Y\).   

Una homotopía entre funciones \(f,g\colon X \to Y\) es una función \(H\colon [0,1]\times X \to Y\)
tal que \(H\) es continua y \(H(0,\cdot) = f(\cdot)\) y \(H(1,\cdot) = g(\cdot)\). Si
existe una homotopía entre \(f\) y \(g\) denotamos \(f\sim g\).   
Decimos que dos espacios \(X\) y \(Y\) son homotópicamente equivalentes si existen
mapas \(f\colon X \to Y\) y \(g \colon Y \to X\) tal que \(f\circ g \sim Id_X\)
y \(g\circ f \sim Id_Y\). 

\subsection{Complejos Simpliciales}

Un complejo simplicial (geométrico) es una colección de símplices que contiene 
todas las caras e intersecciones de sus elementos. Un \(d\)-símplice se define
como la cáscara convexa \(\conv \lbrace v_0, \ldots, v_d\rbrace\) de \(d+1\) puntos
llamados vértices que son 
afinmente independientes. La cáscara convexa de un conjunto \(C\subset\R^n\) es
el subespacio convexo más chico (en el sentido de subconjunto) que contiene
a \(C\). Se dice que \(v_0, \ldots, v_d \in \R^n\) son afínmente independientes
si los vectores relativos \(v_1 -v_0, v_2-v_0, \ldots,v_d-v_0\) son linealmente
independientes. Se dice que \(\tau\) es una cara de un \(d\)-símplice \(\sigma\) 
si es un \(k\)-símplice (\(k<d\)) cuyos vértices son todos vértices de \(\sigma\).

El espacio subyancente de un complejo simplicial \(\KK\) es la unión de todos
sus símplices, denotado \(\bigcup \KK \coloneqq \bigcup_{\sigma\in\KK} \sigma\). 

Un complejo simplicial (abstracto) \(\KK\) es una colección finita de conjuntos
tal que para todo \(\sigma\in\KK\) se tiene que \(\tau\in \KK\) para todo \(\tau\subset\sigma\).
Los elementos de un complejo simplicial se dicen símplices. La dimensión de un símplice
corresponde a su cantidad de elementos menos uno i.e. \(\dim\sigma = \card \sigma-1\). 
La dimensión de \(\varnothing\) es \(-1\). Usualmente vamos a tratar con complejos
simpliciales abstractos cuyos vértices están en \(\R^n\). 

Una triangulación de un espacio topológico \(X\) es una tupla \((\KK,\iota)\) donde
\(\KK\) es un complejo simplicial y \(\iota\colon \bigcup \KK \to X\) es un homeomorfismo.  

\subsection{Geometría Diferencial}

Una \(d\)-variedad es un espacio topológico hausdorff (que separa puntos) y paracompacto
(que todo cubrimiento tiene un subcubrimiento localmente finito) que localmente se asemeja
a \(\R^d\). Vale decir, para todo punto \(x\in X\) existe un \(U\in X\) abierto vecindad de \(x\) 
tal que \(U\approx V\) donde \(V\) es un abierto en \(\R^d\).   

Una \(d\)-variedad \(\MM\) tiene una estructura diferencial si existen homeomorfismos
\(\phi_{\alpha} \colon U_{\alpha} \to V_{\alpha}\) tal que \(\MM = \bigcup_{\alpha} U_{\alpha}\)
y los mapas de transición \(\phi_{\alpha}\circ\phi_{\beta}^{-1}\) son diferenciables.
A la colección de homeomorfimos \(\lbrace \phi_{\alpha} \rbrace_{\alpha}\) se le 
llama atlas. La misma variedad puede tener distintos atlases, por ello, una \(d\)-variedad
diferenciable es una tupla \((\MM, A)\) donde \(A\) es un atlas diferenciable.

Una \(d\)-variedad riemanniana es una tupla \((\MM,g)\) donde \(\MM\) es un \(d\)-variedad topológica
y \(g = \lbrace g_p \rbrace_{p\in \MM}\) es una collección de productos internos defindos positivos
actuando sobre el espacio tangente de un punto \(p\in \MM\).
\begin{displaymath}
  g_p \colon T_p \MM \times T_p \MM \to \R_{\ge 0}.
\end{displaymath}
A \(g\) se le dice métrica riemanniana o tensor de riemann. 

\section{Complejos Fundamentales}

\subsection{Complejo de Delaunay}

Sea \(\X = \left\{ x_1, \dots, x_k \right\} \in \R^n\) un conjunto de puntos en posición
general. Definamos para cada \(x_i\in \X\) el conjunto
\begin{displaymath}
  V_i(\X) \coloneqq \lbrace x\in \R^n \colon \dist(x,x_i) \le \dist(x,x_j)\quad \forall i\ne j\rbrace.
\end{displaymath}
de los puntos más cercanos a \(x_i\) que a cualquier otro punto de \(\X\). Si no enumeramos
los puntos simplemente escribimos \(V_{x}(\X)\) y omitimos \(\X\) si es claro del contexto.  
Denotamos por \(\bigcap V_{\Y} \coloneqq \bigcap_{y\in \Y} V_{y}\) a la región asociada
a \(\Y\subset \X\). Las regiones \(\bigcap V_{\Y}\) se conocen como objetos de Voronoi.

\begin{Definicion}[Complejo de Delaunay]\label{def:delaunay}
  Decimos que \(V = \left\{ v_0, \ldots, v_d \right\} \subset \X\) es un símplice de Delaunay
  si 
  \[
    V_{v_0}(\X) \cap V_{v_1}(\X) \cap \ldots \cap V_{v_d}(\X) \ne \varnothing.
  \]
  El complejo de Delaunay se compone de todos los símplices de Delaunay y se denota
  \(\Del(\X)\).
\end{Definicion}

Una caracterización útil de los símplices de Delaunay, es que la bola
que los circumscribe no contiene en su interior otros puntos de la muestra.

\begin{Proposicion}
  \(\Y\in\Del(\X)\) si y solo si existe una \(n\)-bola cerrada \(B\) tal que 
  \(\Y\subset \partial B\) y \(\X\cap B^{\circ} = \varnothing\).  
\end{Proposicion}
\begin{Demostracion}
  \framebox{\(\Rightarrow\):}
  Sea \(\Y = \lbrace y_0, \dots, y_d \rbrace\). Luego, existe \(y\in\R^n\)
  tal que \(r\coloneqq \dist(y,y_j) = \dist(y,y_i)\) para todo \(0\le i,j \le d\).  
  Luego, \(B = B(y,r)\) por construcción satisface que \(\Y \subset \partial B\).
  Supongamos que \(x\in \X\cap B^{\circ} \ne \varnothing\). Luego, 
  \(\dist(x,y) < r\) y por lo tanto \(y\not\in \bigcap V_{\Y}\). Contradicción.  

  \framebox{\(\Leftarrow\):} El centro de la bola está a la misma distancia de \(\Y\),
  por lo tanto \(\bigcap V_{\Y} \ne \varnothing\). 
\end{Demostracion}

Con esta caracterización podemos probar el siguiente resultado bien conocido.

\begin{Teorema}
  \(\Del(\X)\) es una triangulación de \(\conv\X\).  
\end{Teorema}
\begin{Demostracion}
  Pongamos \(\iota = Id_{\R^n}\).

  \framebox{\(\Rightarrow\):} Sea \(x\in \bigcup \Del(\X)\). Luego, existe \(\Y\in \Del(\X)\) tal que
  \(x\in \conv\Y\) y por lo tanto \(x\in \conv \X\).  

  \framebox{\(\Leftarrow\):} Sea \(x\in \conv \X\). Separamos dos casos:
  \begin{itemize}
   \item Caso 1: \(\card\X \le N+1\): Entonces existe un única bola que circumscribe a \(\X\) y 
     por lo tanto \(\X\in\Del(\X)\).
   \item Caso 2: \(\card\X > N+1\): Tomar \(\Y = \lbrace y_1, \ldots, y_d \rbrace\) los \(d\) 
     puntos más cercanos a \(x\) ordenados por distancias crecientes. Notar \(B(y_d,\dist(x,y_d)) \cap \X \subset \Y\).
     Sea \(y_{d+1}\) el siguiente punto más lejano a \(x\). 
     Luego, \(x\in \conv (y_1, \dots, y_{N+1})\) y existe una única
     esfera que circumscribe a \(y_1, \dots, y_{N+1}\) porque los puntos están en posición
     general.
  \end{itemize}
\end{Demostracion}

Dado \(\MM \subset \R^n\) una superficie (con o sin frontera), denotamos
por \(\Del_{\MM}(\X)\) al complejo formado por aquellos símplices cuyos objetos
de Voronoi asociados que intersectan a \(\MM\). En símbolos:
\begin{displaymath}
  \Del_M(\X)
  \coloneqq
  \left\{ 
    V \subset \X \colon
    \bigcap_{v\in V} V_{v} \cap \MM \ne \varnothing
  \right\}.
\end{displaymath}

Decimos que \(\X\) tiene la propiedad de la intersección genérica para
una \(d\)-variedad sin borde \(X\)
si para todo \(\Y \subset \X\) el polihedro convexo \(P = \bigcap V_{\Y}\) 
satisface:
\begin{enumerate}
  \item \(X \cap P\) es vacío ó
  \item \(X \cap P\) tiene dimensión \(d+\dim P-n\) y \(X \cap P^{\circ} = (X\cap P)^{\circ}\). 
\end{enumerate}
\(\X\) tiene la propiedad de intersección genérica para \(\MM\)  
si se cumple lo anterior con \(X=\MM\) y \(X=\partial \MM\) si \(\MM\) tiene borde.   

Decimos que \(\X\) tiene la propiedad de la bola cerrada con \(\MM\) si
para cualquier \(\Y\subset\X\) se satisface que:
\begin{enumerate}
  \item \(\bigcap V_{\Y} \cap \MM\) es vacía o (homeomorfo) una \((d-\dim\Y)\)-bola cerrada y
  \item \(\bigcap V_{\Y} \cap \partial\MM\) es vacía o (homeomorfo) a una \((d-\dim\Y - 1)\)-bola cerrada.
\end{enumerate}

\begin{Teorema}[Teorema 4.3 de~\cite{edelsbrunnershah}]
  Sea \(\MM\) una \(d\)-variedad compacta en \(\R^n\) con o sin borde y \(\X\)
  una muestra que tiene la propiedad de la intersección genérica
  y la propiedad de la bola cerrada con \(\MM\). Entonces \(\bigcup \Del_{\MM}(\X) \approx \MM\). 
\end{Teorema}
\begin{Demostracion}
  La demostración va por inducción. En cada paso establecemos un homeomorfismo
  y luego lo extendemos.

  Definamos los conjuntos \(B_i \coloneqq \lbrace \bigcap V_{\Y} \cap \MM \colon \dim \Y = d - i\rbrace\).
  Por la propiedad de la bola cerrada, todo elemento de \(B_i\) es (homeomorfo a) una \((d-\dim\Y = i)\)-bola.
  
  Sea \(N = 2^{\card\X}\) e identifiquemos cada \(\Y\subset \X\) con un vector canónico \(e_{\Y} \in \R^N\).

  \underline{Caso Base:} Definamos \(\KK_0 \coloneqq \left\{ e_{\Y} \colon \Y\subset\X, 
  \dim\Y = d, \bigcap V_{\Y}\cap \MM \ne \varnothing\right\}\) y consideremos
  el mapa \(f_0 \colon B_0 \to \KK_0\) dado por \(\bigcap V_{\Y} \cap \MM \mapsto e_{\Y}\).

  \underline{Paso Inductivo:} Sea \(1 \le j \le d-1\).
\end{Demostracion}

\subsection{Complejo de Cech y Rips}

\begin{Definicion}[Complejo de Cech]
  Sea \(\epsilon > 0\). \(V = \lbrace v_0, \dots, v_d \rbrace \subset \X\) es un \(d\)-símplice
  de Cech a escala \(\epsilon\) si
  \[
    B_{\epsilon}(v_0) \cap \ldots \cap B_{\epsilon}(v_d) \ne \varnothing.
  \]
  El complejo de Cech se compone de los símplices de Cech y se denota
  por \(\Cech_{\epsilon}(\X)\). 
\end{Definicion}

Otras caracterizaciones útiles son:
\begin{enumerate}
  \item Un símplice está en el complejo de Cech a escala \(\epsilon\) si y solo si la bola de cobertura mínima
  de sus vértices tiene radio menor que \(\epsilon\). 

  \item El complejo de Cech se corresponde con el nervio de \(\X^{\oplus \epsilon}\). 
\end{enumerate}

La última de las caracterizaciones nos permite aplicar el teorema del nervio
y nos da que \(\Cech_{\epsilon} \simeq \bigcup_{x\in\X} B_{\epsilon}(x)
= \X^{\oplus\epsilon}\), así que queda ver si \(\X^{\oplus\epsilon}\simeq \MM\).

\begin{Teorema}[Proposición 3.2 de~\cite{smale}]
  Sea \(\MM\) es una \(d\)-variedad riemanniana compacta y suave. Sea \(\X\) una muestra
  \(\epsilon\)-densa y \(\epsilon\)-ruidosa de \(\MM\) con \(\epsilon < \sqrt{3/5} \reach \MM\).
  Entonces
  \(\MM\) es un retracto por deformación de \(\X^{\oplus \epsilon}\). 
  En particular, \(\X^{\oplus\epsilon}\simeq \MM\).   
\end{Teorema}
\begin{Demostracion}
  Notar que para \(\epsilon < \reach\MM\) la proyección \(\pi\colon\X^{\oplus\epsilon} \to \MM\)
  está bien definida. Mostraremos que el retracto viene dado por
  \begin{displaymath}
  \begin{aligned}
    H\colon [0,1]\times\X^{\oplus\epsilon} &\to \MM\\
    (t,x) &\mapsto t\pi(x) + (1-t)x
  \end{aligned}.
  \end{displaymath}
  para \(\epsilon < \sqrt{3/5}\reach\MM\). 

  Para esto debemos probar que para todo \(x\in\X^{\oplus\epsilon}\), el 
  segmento dado por \(H(\cdot,x)\) está completamente contenido en \(\X^{\oplus\epsilon}\).

  Notése que si \(m\in\MM\), entonces \(\pi^{-1}(m) = (T_m\MM)^{\perp} \cap \X^{\oplus\epsilon} \cap B_{\reach\MM}(m)\).
  Por lo tanto, basta revisar los \(x\in\pi^{-1}(m)\). Distinguimos dos casos:

  \(\bullet\)~\(x\in B_{\epsilon}(x_i)\cap\pi^{-1}(m)\) para algún \(x_i\in\X\cap B_{\epsilon}(m)\). Entonces
  el segmento de \(x\) a \(m\) está completamente contenido en \(B_{\epsilon}(x_i)\) y tenemos el resultado.

  \(\bullet\)~\(x\in B_{\epsilon}(x_i)\cap\pi^{-1}(m)\) para algún \(x_i\in\X\cap B_{\epsilon}(m)^c\).
  Notar que \(x_i\in (T_m^{\perp})^{\oplus\epsilon}\cap B_{\epsilon}(m)^c\cap B_{\reach\MM}(m)\).
  Considere las bolas tangentes a \(m\) de radio \(\reach\MM\) a cada lado del plano tangente \(T_m\). 
  Llamemoslas \(B_{+}\) y \(B_{-}\). 
  Luego, la vecindades de \(m\) en la superficie viven en \(B_{+}^c \cap B_{-}^{c}\) (de lo contrario
  el \(\reach\MM\) podría haber sido menor, lo cual es una contradicción). De esta forma, 
  \(x_i\) debe estar en el borde de \(B_{+}\) (o \(B_{-}\), el argumento es simétrico)
  para que la distancia entre \(x\) y \(m\) se maximize. 

  En resumen, la distancia entre \(x\) y \(m\) se maximiza cuando \(x_i\) vive en la intersección
  entre el borde de \((T_m^{\perp})^{\oplus\epsilon}\) y el borde de \(B_{+}\). Queda ver cuál
  es el máximo valor de \(\epsilon\) que garantiza que \(H(\cdot,x) \in \X^{\oplus\epsilon}\). 

  \begin{wrapfigure}[7]{r}{0.3\textwidth}
  \vspace{-0.7cm}
    \centering
    \begin{tikzpicture}
      \draw (-2,0) -- (2,0) node[right] {\(T_m\)}; 
      \draw (0,2) circle(2cm);
      \draw[thick] (0,0) -- (0,2) node[above] {\small\(\reach\MM\)};
      \draw (0,1.7) node[left] {\(x\)};
      \draw (0,0) node[below] {\(m\)};
      \draw (0,1.7) -- +(-20:1.9cm) node[midway,above] {\(\epsilon\)} node[right] {\(x_i\)};
      \draw (0,0) -- ([yshift=1.7cm] -20:1.9cm) node[midway,below] {\(\theta\)};
    \end{tikzpicture}
  \end{wrapfigure}

  Sea \(A\) la distancia entre \(x\) y \(m\), \(b\) la distancia entre \(m\) y \(x_i\) y
  \(\theta\) el ángulo entre \(T_m\) y \(x_i\). Notar que \(m\) y \(x_i\) está a distancia
  \(\reach\MM\) del centro \(c_{+}\) de \(B_{+}\). Luego, el triángulo \((c_{+},m,x_i)\) es isóceles
  y por lo tanto \(b = 2\reach\MM \sin(\alpha)\) donde \(\alpha\) es la mitad del ángulo \((m,c_{+},x_i)\).
  Como el ángulo \((x_i,m,c_+)\) es \(\pi/2-\theta\), se sigue que \(\alpha = \theta\) puesto que
  el ángulo desde \(m\) al punto medio de \(b\) y \(c_+\) es \(\pi/2\). Por lo tanto,
  \(b = 2\reach\MM\sin(\theta)\). Se sigue que
  \begin{alignat*}{1}
    A 
    &= b\sin(\theta) + \sqrt{\epsilon^2 - b^2\cos^2(\theta)}\\
    &= 2\reach\MM\sin^2(\theta) + \sqrt{\epsilon^2 - 4\reach\MM^2\sin^2(\theta)\cos^2(\theta)}
  \end{alignat*}
  Podemos ver \(A\) como una función de \(\theta\) y optimizarla. Esto nos deja con

\end{Demostracion}

El complejo de Cech es caro de computar, por ello, se suele usar el complejo de Rips,
que relaja las condiciones.

\begin{Definicion}[Complejo de Rips]
  Sea \(\epsilon > 0\). \(V = \lbrace v_0, \dots, v_d \rbrace \subset \X\) es un \(d\)-símplice
  de Rips a escala \(\epsilon\) si
  \begin{displaymath}
    \dist(u,v) \le \epsilon \quad \forall u,v \in V.
  \end{displaymath}
  El complejo de Rips se compone de los símplices de Rips y se denota
  por \(\Rips_{\epsilon}(\X)\).
\end{Definicion}

Bajo algunas condiciones, el complejo de Rips se parece al complejo de Cech.
\begin{Proposicion}[Teorema 2.5 de~\cite{desilva}]
  Sea \(\X\subset \R^n\) finito y \(\epsilon > 0\). Sea 
  \(\rho_n \coloneqq \sqrt{\frac{n}{n+1}}\). Luego,
  \begin{displaymath}
    \Cech_{\epsilon}(\X) \subset \Rips_{2\epsilon}(\X) \subset \Cech_{2\epsilon \rho_n}(\X)
  \end{displaymath}
\end{Proposicion}
\begin{Demostracion}
  Para la inclusión de la izquierda: Sea \(\lbrace v_0, \dots, v_d \rbrace\) un símplice
  de Cech. Sea \(x\in \bigcap B_{\epsilon}(v_i) \ne \varnothing\). Luego,
  \(\dist(v_i,v_j) \le \dist(v_i, x) + \dist(x, v_j) \le 2\epsilon\). 

  Para la inclusión de la derecha: Sea \(V = \lbrace v_0, \dots v_d \rbrace\) un símplice de Rips.
  Supongamos el peor caso en que \(\dist(v_i,v_j) = 2\epsilon\) para todo \(i\ne j\). 
  Luego, \(V\) se puede identificar el símplice en \(\R^{d+1}\) formado por 
  los vectores \(\lbrace 2\epsilon\cdot e_1, \dots, 2\epsilon\cdot e_{d+1}\rbrace \subset \R^{d+1}\). 
  La bola que circumscribe a \(V\) tiene como centro 
  \(c = \frac{2\epsilon}{d+1} (e_1 + \cdots + e_{d+1}) = \frac{2\epsilon}{d+1} (1,\dots,1)\).
  Aplicando pitágoras
  \begin{displaymath}
    r_{d} \coloneqq 
    \sqrt{\norm{2\epsilon\cdot e_i}^2 - \norm{c}^2} 
    =
    \sqrt{(2\epsilon)^2 - \frac{(2\epsilon)^2}{d+1}}
    =
    2\epsilon \sqrt{\frac{d}{d+1}}
    =
    2\epsilon \rho_d
  \end{displaymath}
  vemos que la bola de cobertura mínima tiene radio \(r_{d}\) y por lo tanto \(V\) es un 
  símplice de Cech a escala \(r_{d}\). Notando que \(\rho_d \le \rho_n\) para \(d+1 \le n\)
  y que no hay \(n+1\) símplices en \(\R^n\) se sigue el resultado.
\end{Demostracion}

\begin{thebibliography}{9}
\bibitem{desilva}
De Silva, V., \& Ghrist, R. (2007). Coverage in sensor networks via persistent homology. Algebraic \& Geometric Topology, 7(1), 339-358.

\bibitem{edelsbrunnershah}
Edelsbrunner, H., \& Shah, N. R. (1994, June). Triangulating topological spaces. In Proceedings of the tenth annual symposium on Computational geometry (pp. 285-292).

\bibitem{smale}
Niyogi, P., Smale, S., Weinberger, S. (2008). Finding the homology of submanifolds with high confidence from random samples. Discrete \& Computational Geometry, 39, 419-441.

\bibitem{welzl}
Welzl, E. (2005, June). Smallest enclosing disks (balls and ellipsoids). In New Results and New Trends in Computer Science: Graz, Austria, June 20–21, 1991 Proceedings (pp. 359-370). Berlin, Heidelberg: Springer Berlin Heidelberg.

\end{thebibliography}

\appendix

\section{Más sobre Delaunay}

El complejo de Delaunay se puede ver como un caso particular de un problema de optimización.
Sea \(U\) un dominio (abierto conexo y acotado) en \(\R^n\) triangulable. 
Sea \(f\colon U \to \R\) una función continua. El error de interpolación lineal
en norma \(p\in [1,\infty]\) de una triangulación \(T\) de \(U\) se define como:
\begin{displaymath}
  E_p(U,T,f) \coloneqq \norm{f - L_f}_{p}
\end{displaymath}
donde \(L_f\) es una interpolación lineal de \(f\) en \(T\).

\begin{Teorema}%[\cite{chenoptimal}]
  Sea \(\X\) un conjunto finito de \(\R^n\).  
  El complejo de Delaunay definido en~(Definición \ref{def:delaunay}) es el que minimiza el error de interpolación
  de la función \(\norm{\cdot}^2 \colon \conv \X \to \R\) entre todos los complejos simpliciales
  con conjunto de vértices en \(\X\). 
  \begin{displaymath}
    E_p(\conv\X, \Del(\X), \norm{\cdot}^2)
    =
    \argmin_{T} E_p(\conv\X, T, \norm{\cdot}^2).
  \end{displaymath}
\end{Teorema}

\section{Bola de Cobertura Mínima}

\end{document}
